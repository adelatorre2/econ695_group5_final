\section*{Conclusion}
\addcontentsline{toc}{section}{Conclusion}
\label{sec:Conclusion}


Our analysis began with a simple descriptive comparison of men and women and revealed a substantial raw gender wage gap: women earn markedly lower wages, work with lower-paid coworkers, and are distributed differently across education categories. These patterns motivated a deeper decomposition of the wage gap into components attributable to observable characteristics and components arising from differential returns.

Standard wage models and the Oaxaca decomposition showed that differences in education and experience explain relatively little of the gender wage gap. Despite women having slightly  higher formal schooling on average, men receive systematically higher returns to education  and experience. Most of the gap is therefore ``unexplained'' in the Oaxaca sense—arising from differences in how observable characteristics are rewarded, as well as factors not captured  in the model. This early finding established a consistent theme: the gap is driven less by who men and women are, and more by how the labor market rewards them.

Introducing coworker wages into the analysis added an important new dimension. We found that the average wage of one's coworkers is a powerful predictor of individual wages, and that men both sort into higher–wage coworker environments and receive larger returns to these environments. Yet even after accounting for these differences in exposure, coworker wages explain only a small share of the overall wage gap. Most of the gap remains tied to differences in returns, reinforcing the conclusion that workplace environments are not gender–neutral.

The event–study evidence allowed us to probe the mechanisms behind these patterns. Workers who move into higher–wage coworker environments experience immediate and persistent increases in their own wages, while downward moves produce smaller or negative changes. These discontinuities are hard to reconcile with a pure ability–sorting model, since they reflect sharp shifts in coworker wage exposure at the moment of a job transition. Instead, the results support a mixed interpretation: access, networks, or search behavior affect  who reaches high–wage workplaces, while genuine spillovers or institutional pay norms help sustain wage differences once workers arrive.

Our first–difference models strengthened this conclusion by differencing out time–invariant worker characteristics. Roughly half of the strong OLS association between coworker wages  and own wages survives in the differenced specification. This suggests that sorting on fixed  traits explains part—but not all—of the relationship. Workers’ wages do respond to changes  in coworker wage environments, but the size of this response is modest and heterogeneous.

Finally, our shrinkage exercises confirmed that the OLS and Ridge first–difference estimates are numerically stable, while the Lasso's aggressive shrinkage toward zero highlighted the fragility of the average coworker–wage effect once heavy penalization is imposed. Together,  these results suggest that coworker wages matter for wage growth, but their influence is far  from overwhelming and depends on model flexibility.

Taken together, our findings reveal meaningful progress in understanding the gender wage gap. The gap is shaped by a combination of sorting into different workplace environments, differential  returns to those environments, and deeper structural differences in how men’s and women’s characteristics are rewarded. Coworker wages help illuminate these mechanisms, but they do not eliminate the gap. Instead, they clarify that gender differences in wage setting arise not only from observable characteristics but also from workplace dynamics, institutional structures, and unobserved factors that disproportionately benefit men.

Overall, the evidence implies that the gender wage gap remains persistent even after carefully controlling for education, experience, job transitions, and coworker environments. While our analysis cannot definitively isolate all underlying mechanisms, it demonstrates that both sorting and differential returns play central roles. Future work incorporating richer measures of job  tasks, bargaining power, and organizational policies would provide an even more complete account of how gender shapes wages in the labor market.
\newpage