\section*{Data}

In this section, we describe the dataset used in our analysis and provide descriptive statistics that summarize the main characteristics of workers in our sample. The data come from administrative records from Country~X, a labor market that closely resembles the United States \citep{wuFinalProjectAnalysis2025}. Each observation corresponds to a single individual who is observed for multiple consecutive years across two distinct jobs. All individuals in the dataset are observed for at least three years on their first job (periods $-3$, $-2$, $-1$) and three years on their second job (periods $0$, $1$, $2$).

\subsection*{Sample Size and Structure}

The dataset contains a total of 16{,}969 individuals, of whom 10{,}575 are men and 6{,}394 are women. Thus, approximately 38\% of the sample consists of women. Each individual record includes demographic characteristics, education, experience, and a sequence of wage observations around a job transition.

The timing convention follows the project instructions: period $0$ is the first year on the second job, periods $1$ and $2$ are the subsequent years, and periods $-1$ through $-3$ correspond to the final three years on the first job. For each job spell, the dataset also includes a measure of the mean log wage of coworkers, which is central to our later analysis of coworker wage effects.

\subsection*{Variables}

The main variables used in our analysis are:

\begin{itemize}
    \item \textbf{Demographics:} age, years of education (educ), and potential experience (exp).
    Education takes one of four values: 6, 9, 12, and 16 years.
    \item \textbf{Gender:} a binary indicator variable \texttt{female} equal to 1 for women and 0 for men.
    \item \textbf{Wages:} log hourly wages measured in different periods:
    \begin{itemize}
        \item $y$: wage in period 0 (first year on the second job),
        \item $yp1$ and $yp2$: wages in periods 1 and 2,
        \item $yl1$, $yl2$, $yl3$: wages in periods $-1$, $-2$, and $-3$ on the first job.
    \end{itemize}
    \item \textbf{Coworker Wages:} mean log wages of coworkers at:
    \begin{itemize}
        \item the first job ($owage1$), averaged over periods $-1$ and $-2$,
        \item the second job ($owage2$), averaged over periods 0, 1, and 2.
    \end{itemize}
\end{itemize}

\subsection*{Descriptive Statistics}

Table~\ref{tab:summarystats} reports descriptive statistics for the main variables in the dataset.
Women in the sample tend to have slightly lower wages, both on their first and second jobs, and
the distribution of education levels differs meaningfully between men and women. Consistent
with the project instructions, experience ranges between 5 and 30 years.

% ------------------------------
% Summary statistics table
% ------------------------------

\begin{table}[H]
\centering
\caption{Summary Statistics}
\label{tab:summarystats}
\begin{tabular}{lcccc}
\toprule
Variable & Mean & Std.~Dev. & Min & Max \\
\midrule
y (log wage, period 0) & 1.79 & 0.65 & 0.60 & 4.34 \\
age                    & 33.56 & 5.69 & 22 & 52 \\
educ (years)           & 10.48 & 3.59 & 6 & 16 \\
female                 & 0.38 & 0.49 & 0 & 1 \\
exp                    & 17.07 & 6.45 & 5 & 30 \\
yl1                    & 1.74 & 0.61 & 0.60 & 4.24 \\
owage2                 & (fill from code) & & & \\
\bottomrule
\end{tabular}
\end{table}

\subsection*{Gender Differences in the Sample}

Women represent 38\% of the sample and differ systematically from men in several observable
characteristics. Women have slightly lower wages on average and are distributed differently
across the four education categories. These patterns, visible in the value counts and summary
statistics from our data, motivate our analysis of the gender wage gap in the sections that follow.

In Figure~\ref{fig:wagedist}, we plot the distribution of log wages by gender. Consistent with our
summary statistics, women have a lower wage distribution than men. This motivates the
subsequent regression and decomposition analyses.

% Figure placeholder
\begin{figure}[H]
\centering
\includegraphics[width=0.75\textwidth]{fig_distribution_wages}
\caption{Distribution of Log Wages by Gender}
\label{fig:wagedist}
\end{figure}