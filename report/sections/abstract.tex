\begin{abstract}
We use linked employer–employee administrative data to study the sources of the gender wage gap and the role of coworker wage environments in shaping wage growth. We document descriptive differences between men and women in wages, education, experience, and exposure to higher–paid coworkers. We estimate wage regressions and Oaxaca decompositions to separate the gap into components due to observable characteristics and differential returns, and we conduct event–study analyses of wage dynamics around job moves to identify how changes in coworker wage environments affect individual wage adjustments. Finally, we estimate first–difference models with OLS, Ridge, and Lasso to assess the robustness of these relationships. Our results show that most of the gender wage gap arises from differences in returns rather than differences in characteristics, that workers experience meaningful (but heterogeneous) wage changes when moving to higher– or lower–wage coworker environments, and that penalized estimators highlight which predictors are most stable. These findings advance our understanding of how workplace sorting and differential returns jointly contribute to the persistence of the gender wage gap.
\end{abstract}