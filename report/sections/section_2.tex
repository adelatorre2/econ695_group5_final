\section*{Section 2: Gender Wage Gaps}
\addcontentsline{toc}{section}{Section 2: Gender Wage Gaps}
\label{sec:Section 2: Gender Wage Gaps}

From our descriptive analysis, it is evident that there exists substantial raw differences in wages, education, and coworker wage environments between men and women. These gaps raise a natural question: to what extent do observable characteristics, such as education and potential experience, account for the wage differences we see, and how much remains unexplained by these factors? We begin to attempt to investigate these questions by estimating standard wage models and applying the Oaxaca decomposition to separate the portion of the gender wage gap attributable to differences in characteristics from the portion attributable to differences in returns to those characteristics.

\subsection*{2.1 Standard Models and Oaxaca Decomposition}
\addcontentsline{toc}{subsection}{2.1 Standard Models and Oaxaca Decomposition}
\label{subsec:2.1 Standard Models and Oaxaca Decomposition}

\begin{table}[H]
\centering
\caption{Wage Models and Oaxaca Decomposition of the Gender Wage Gap}
\label{tab:table2}
\begin{threeparttable}
\footnotesize
\renewcommand{\arraystretch}{0.8}
\begin{tabular}{lcccc}
\toprule
 & (1) & (2) & (3) & (4) \\
 & Female Dummy & Full Model & Men & Women \\
\midrule
Intercept      & 1.866 & 0.557 & 0.491 & 0.327 \\
               & (0.006) & (0.056) & (0.078) & (0.080) \\
Female         & -0.208 & -0.271 &  &  \\
               & (0.010) & (0.007) &  &  \\
C(educ)[T.9]   &  & 0.275 & 0.259 & 0.314 \\
               &  & (0.009) & (0.012) & (0.015) \\
C(educ)[T.12]  &  & 0.680 & 0.692 & 0.667 \\
               &  & (0.009) & (0.012) & (0.014) \\
C(educ)[T.16]  &  & 1.515 & 1.508 & 1.525 \\
               &  & (0.011) & (0.014) & (0.016) \\
exp            &  & 0.067 & 0.072 & 0.069 \\
               &  & (0.011) & (0.015) & (0.015) \\
exp2           &  & -0.001 & -0.001 & -0.001 \\
               &  & (0.001) & (0.001) & (0.001) \\
exp3           &  & -0.000 & -0.000 & 0.000 \\
               &  & (0.000) & (0.000) & (0.000) \\
\midrule
R-squared     & 0.024 & 0.588 & 0.556 & 0.618 \\
Adj.\ R$^{2}$ & 0.024 & 0.588 & 0.556 & 0.618 \\
N             & 16{,}969 & 16{,}969 & 10{,}575 & 6{,}394 \\
\bottomrule
\end{tabular}

\begin{tablenotes}
\footnotesize
\item \textit{Notes}: Columns (1)–(2) report pooled OLS models; Columns (3)–(4) estimate the same specification separately by gender. Standard errors in parentheses. Oaxaca decomposition uses gender-specific models.
\end{tablenotes}

\end{threeparttable}
\end{table}

In the pooled regression that includes a constant and only a female dummy, the coefficient on $female = -0.2084$ indicates that, holding no other variables constant, women earn about $20.8$\% lower wages than men on average. This is representative of the fundamental gender wage gap. On adding categorical education and a cubic polynomial in experience as controls to our pooled regression, the coefficient on $female = -0.2706$ which reveals that on controlling for education and experience, women earn approximately $27.06$\% less than men on average. Since the coefficient becomes increasingly negative on adding the respective controls, we interpret that women in our sample are on average more educated and experienced than men. Thus controlling for these variables widened the wage gap.

In the separate by gender model, the coefficients on education split categorically and on experience are positive and significant for men and women alike. As education levels increase, returns to education on wages rise sharply, while the returns to experience on wages seem to show diminishing returns with increase in experience. On comparing the two groups, we conclude that men have much higher intercept values at $0.4906$ compared to $0.3270$ for women. This indicates that the predicted log wage for men with 6 years of education and no experience is $0.4906$ whereas it is $0.3270$ for women. While the returns to education are quite similar for both men and women, men have a marginally higher return to experience. This means that controlling for other variables, for a unit increase in years of experience, log wage rose by $0.0723$ for men and $0.0686$ for women. 

While these regressions quantify the wage gap and its relationship to observable traits, part of the wage gap may be due to differences in the average education and experience levels (composition effects), while another being due to differences each group being paid differently for the same education or experience, or actual gender discrimination, which is the object of our study. 

To separate the portion of the wage gap arising from differences in average characteristics from the portion arising from differences in how those characteristics are rewarded in the labor market, we apply the Oaxaca decomposition of \citet{oaxacaMaleFemaleWageDifferentials1973} using the gender-specific regressions in Columns (3) and (4). Let $y_{gi}$ denote log hourly wages for individual $i$ in group $g \in \{m,f\}$ and let $X_{gi}$ be the vector of observable characteristics (education dummies and a cubic
polynomial in potential experience).\footnote{The intercept is excluded from$X_{gi}$ and treated separately.} We model wages as

\begin{equation}
  y_{gi} = \alpha_g + X_{gi}'\beta_g + \varepsilon_{gi},
  \qquad g \in \{m,f\}.
\end{equation}

Taking sample means yields
\[
  \bar y_g = \hat\alpha_g + \bar X_g' \hat\beta_g.
\]
The raw gender wage gap is defined as
\[
  \Delta \equiv \bar y_m - \bar y_f.
\]

Using women as the reference group, the Oaxaca decomposition expresses
$\Delta$ as
\begin{equation}
\label{eq:oaxaca}
  \Delta 
  = (\bar X_m - \bar X_f)' \hat\beta_f 
    \;+\; \bar X_m'(\hat\beta_m - \hat\beta_f).
\end{equation}

The first term,
\[
  (\bar X_m - \bar X_f)' \hat\beta_f,
\]
is the \emph{explained} or \emph{between-group} component: it reflects how
differences in average education and experience across men and women would
translate into a wage gap if both groups received the same returns estimated
for women.  

The second term,
\[
  \bar X_m'(\hat\beta_m - \hat\beta_f),
\]
is the \emph{unexplained} or \emph{within-group} component: it captures
differences in the returns to these characteristics, evaluated at men’s mean
characteristics.

The Oaxaca Decomposition result shows the total wage gap is approximately $0.045$ log points, meaning men earn about $4.5$\% higher wages than women on average in period $0$. Based on the included education and experience characteristics, the \emph{explained component} $(\text{the between component}) = –0.0623$ indicates that women should earn higher wages than men. That is to say, if the labor market rewarded these characteristics equally for both genders, the wage gap would favor women. This is intuitive as women in the sample have slightly higher educational attainment and comparable experience as displayed above. However, given the unexplained component reflects differences in the returns to characteristics by definition, women are not favored in this case since the \emph{unexplained component} $(\text{the within component}) = 0.1071$ suggests that, even though women possess characteristics associated with higher wages, they do not receive the same returns as men. Ultimately, the Table~\ref{tab:table2} results show a positive wage gap favoring men.



\subsection*{2.2 Gender Difference in Experience Profiles}
\addcontentsline{toc}{subsection}{2.2 Gender Difference in Experience Profiles}\label{subsec:2.2 Gender Difference in Experience Profiles}

The existence of a wage gap we identified in \nameref{subsec:2.1 Standard Models and Oaxaca Decomposition} is not monotone as behaviors, attitudes, lawsuits, macro-trends, and many other of the world's moving parts can influence changes in any such gap we have identified and maybe too effects people differently. As such, it is necessary to ask how it evolves over the life cycle. Here, we explore gender differences in the relationship between log wages and potential experience, focusing first on workers with 12 years of education and then extending the comparison across the full education distribution.

\begin{figure}[H]
    \centering
    \includegraphics[width=0.7\textwidth]{figures/fig2_exp_profile_edu12_vibhu.png}
    \caption{Experience--wage profiles by gender for workers with 12 years of education. Notes: Figure plots log hourly wages against potential experience (5--30 years) and overlays the fitted values from a cubic polynomial in experience, estimated separately for men and women.}
    \label{fig:exp_profile_edu12}
\end{figure}

In the primary Figure~\ref{fig:exp_profile_edu12} plot $(\text{education} = 12$), male wages start above female wages even at low experience levels, and the gap widens as experience increases. The male curve shows a sharper upward slope and peaks at a higher level than the female profile. In contrast, the female curve is flatter which suggests that women receive smaller marginal returns to experience. Both curves show wages grow quickly early in a career and then level off. Notably, concurrent to Figure~\ref{fig:wage_dists}, this effect is stronger for men as the women’s profile peaks earlier and at a substantially lower predicted wage. Together, these patterns visually reinforce the Oaxaca decomposition result that women have similar and even more favorable education levels than men, but the returns to those characteristics differ considerably.

\begin{center}

    %------------------ Top row ------------------%
    \begin{minipage}[b]{0.45\textwidth}
        \centering
        \includegraphics[width=\textwidth]{figures/fig2_bonus_exp_profile_edu6_vibhu.png}\\[0.2em]
        {\footnotesize (a) Educ = 6}
    \end{minipage}
    \hfill
    \begin{minipage}[b]{0.45\textwidth}
        \centering
        \includegraphics[width=\textwidth]{figures/fig2_bonus_exp_profile_edu9_vibhu.png}\\[0.2em]
        {\footnotesize (b) Educ = 9}
    \end{minipage}

    \vspace{0.4cm}

    %------------------ Bottom row ------------------%
    \begin{minipage}[b]{0.45\textwidth}
        \centering
        \includegraphics[width=\textwidth]{figures/fig2_bonus_exp_profile_edu12_vibhu.png}\\[0.2em]
        {\footnotesize (c) Educ = 12}
    \end{minipage}
    \hfill
    \begin{minipage}[b]{0.45\textwidth}
        \centering
        \includegraphics[width=\textwidth]{figures/fig2_bonus_exp_profile_edu16_vibhu.png}\\[0.2em]
        {\footnotesize (d) Educ = 16}
    \end{minipage}

    \vspace{0.5cm}

    {\footnotesize
    \textit{Bonus Figure:} Experience--wage profiles by gender and education level. 
    Each panel plots log hourly wages against potential experience for the indicated 
    education category and overlays gender-specific cubic fits. This bonus panel is 
    intentionally unnumbered and does not contribute to the main sequence of figures.
    }

\end{center}

\noindent
The bonus plots for education levels 6, 9, 12, and 16 reveal that the pattern holds across the entire education distribution. At all education levels, male predicted wages exceed female predicted wages for the same experience. Moreover, male returns to experience are consistently larger as the slope of the male curve is steeper in the early career years and male profiles peak at higher levels of log wages which strengthens the results from Figure~\ref{fig:wage_dists}and Table~\ref{tab:table2}. Thus, even at higher education levels (e.g., 16 years), although both men and women earn more overall, the gender gap persists and grows with experience. In other words, these panels visually reinforce that women begin their careers at a lower baseline and appear to receive systematically weaker returns to experience, even when education is held fixed. 