\section*{Section 2: Gender Wage Gaps}
\addcontentsline{toc}{section}{Section 2: Gender Wage Gaps}
\label{sec:Section 2: Gender Wage Gaps}

From our descriptive analysis, it is self evident that there exists substantial raw differences in wages, education, and coworker wage environments between men and women. These gaps raise a natural question: to what extent do observable characteristics—such as education and potential experience—account for the wage differences we see, and how much remains unexplained by these factors? Here we begin to attempt to formally investigate these questions by estimating standard wage models and applying the Oaxaca decomposition to separate the portion of the gender wage gap attributable to differences in characteristics from the portion attributable to differences in returns to those characteristics.

\subsection*{2.1 Standard Models and Oaxaca Decomposition}
\addcontentsline{toc}{subsection}{2.1 Standard Models and Oaxaca Decomposition}
\label{subsec:2.1 Standard Models and Oaxaca Decomposition}

\begin{table}[H]
\centering
\caption{Wage Models and Oaxaca Decomposition of the Gender Wage Gap}
\label{tab:table2}
\begin{threeparttable}
\footnotesize
\renewcommand{\arraystretch}{0.8}
\begin{tabular}{lcccc}
\toprule
 & (1) & (2) & (3) & (4) \\
 & Female Dummy & Full Model & Men & Women \\
\midrule
Intercept      & 1.866 & 0.557 & 0.491 & 0.327 \\
               & (0.006) & (0.056) & (0.078) & (0.080) \\
Female         & -0.208 & -0.271 &  &  \\
               & (0.010) & (0.007) &  &  \\
C(educ)[T.9]   &  & 0.275 & 0.259 & 0.314 \\
               &  & (0.009) & (0.012) & (0.015) \\
C(educ)[T.12]  &  & 0.680 & 0.692 & 0.667 \\
               &  & (0.009) & (0.012) & (0.014) \\
C(educ)[T.16]  &  & 1.515 & 1.508 & 1.525 \\
               &  & (0.011) & (0.014) & (0.016) \\
exp            &  & 0.067 & 0.072 & 0.069 \\
               &  & (0.011) & (0.015) & (0.015) \\
exp2           &  & -0.001 & -0.001 & -0.001 \\
               &  & (0.001) & (0.001) & (0.001) \\
exp3           &  & -0.000 & -0.000 & 0.000 \\
               &  & (0.000) & (0.000) & (0.000) \\
\midrule
R-squared     & 0.024 & 0.588 & 0.556 & 0.618 \\
Adj.\ R$^{2}$ & 0.024 & 0.588 & 0.556 & 0.618 \\
N             & 16{,}969 & 16{,}969 & 10{,}575 & 6{,}394 \\
\bottomrule
\end{tabular}

\begin{tablenotes}
\footnotesize
\item \textit{Notes}: Columns (1)–(2) report pooled OLS models; Columns (3)–(4) estimate the same specification separately by gender. Standard errors in parentheses. Oaxaca decomposition uses gender-specific models.
\end{tablenotes}

\end{threeparttable}
\end{table}

Across the pooled models, the coefficient on the female indicator is negative and statistically significant, indicating that women earn lower wages on average even before conditioning on observable characteristics. In the baseline model that includes only a constant and the female dummy, the coefficient of approximately $-0.208$ suggests that women earn roughly 20\% lower wages than men at period~0. After adding categorical controls for education and a cubic polynomial in experience, the wage gap widens to about 27\%, reflecting that women in our sample tend to have slightly higher educational attainment and similar experience levels. Thus, once controlling for these favorable characteristics, the remaining wage gap attributable to gender becomes larger.

The separate regressions by gender show broadly similar returns to education and experience for men and women. Returns to schooling rise with each additional educational category, while experience exhibits diminishing marginal returns. The largest difference lies in the intercepts: men have a higher baseline log wage than women, implying that even at comparable education and zero experience, men start from a higher predicted wage level. 


Yet this baseline gap is muddied between differences in the composition of the characteristics of each group (e.g., average education levels) and differences in the returns to those characteristics (e.g., how much additional education translates to wage increases). In other words, while these regressions quantify the wage gap and its relationship to observable traits, part of the wage gap may be due to differences in the average education and experience levels (composition effects), while another being due to differences each group being paid differently for the same education or experience, or actual gender discrimination, which is the object of our study. 

To disentangle these two effects, we apply the Oaxaca decomposition from \citet{oaxacaMaleFemaleWageDifferentials1973} using the gender-specific models from Columns (3) and (4) to formally separate the composition-driven (between) component from the differences in returns (within) component. To model this, let $y_{gi}$ denote log hourly wages in period $0$ for individual $i$ in group $g \in \{m,f\}$ (men, women), and let $X_{gi}$ be the corresponding row vector of observable characteristics (education dummies and a cubic polynomial in potential experience).\footnote{The intercept is kept separate and not included in $X_{gi}$.} The group–specific wage equations are
\begin{equation}
  y_{gi} = \alpha_g + X_{gi}'\beta_g + \varepsilon_{gi}, 
  \qquad g \in \{m,f\}.
\end{equation}
Taking expectations and using sample means,
\[
  \bar y_g = \hat\alpha_g + \bar X_g' \hat\beta_g,
\]
where $\bar y_g$ and $\bar X_g$ are the sample means for group $g$ and
$(\hat\alpha_g,\hat\beta_g)$ are the OLS estimates from the gender–specific
regressions.

We define the (raw) gender wage gap as the difference in mean log wages
between men and women,
\[
  \Delta \equiv \bar y_m - \bar y_f.
\]
Using women as the reference group for the “non–discriminatory’’ wage
structure, the Oaxaca (1973) decomposition writes this gap as
\begin{equation}
\label{eq:oaxaca}
  \Delta 
  = (\bar X_m - \bar X_f)' \hat\beta_f 
    \;+\; \bar X_m'(\hat\beta_m - \hat\beta_f).
\end{equation}
The first term,
\[
  (\bar X_m - \bar X_f)' \hat\beta_f,
\]
is the \emph{explained} or \emph{between–group} component: it captures how
differences in average characteristics (education and experience) between men
and women would translate into a wage gap if both groups were paid according
to the same returns $\hat\beta_f$ estimated for women. The second term,
\[
  \bar X_m'(\hat\beta_m - \hat\beta_f),
\]
is the \emph{unexplained} or \emph{within–group} component: it reflects
differences in the returns to characteristics between men and women, evaluated
at men’s average characteristics. In our implementation, the explained and
unexplained components in equation~\eqref{eq:oaxaca} correspond exactly to the
two pieces reported in the decomposition based on the separate male and female
regressions used for Table~\ref{tab:table2}.



\subsection*{2.2 Gender Difference in Experience Profiles}
\addcontentsline{toc}{subsection}{2.2 Gender Difference in Experience Profiles}\label{subsec:2.2 Gender Difference in Experience Profiles}
