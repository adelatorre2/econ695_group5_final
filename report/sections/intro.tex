\section*{Introduction}
\addcontentsline{toc}{section}{Introduction}
\label{sec:Introduction}

Understanding why men and women continue to earn different wages remains one of the central questions in labor economics. While the raw gender pay gap is well documented, much less is known about how workplace environments, job transitions, and the characteristics of coworkers shape these disparities \citep{blauGenderWageGap2017,brickGenderWageGap2023}. Our dataset offers a rare opportunity to study these mechanisms directly: we observe workers across consecutive jobs, along with detailed information on their coworkers’ average wages. This structure allows us to examine not only individual wage differences, but also the role of sorting, workplace quality, and the returns workers receive in different environments. 

We combine descriptive evidence, standard wage regressions, Oaxaca decomposition, event-study analyses of job moves, and shrinkage methods such as Ridge and Lasso to build a more complete picture of the gender wage gap. Each approach highlights a different mechanism: differences in characteristics, differences in returns, mobility patterns, and model robustness—and together they help us understand how gender shapes the wage trajectories workers experience. Our goal is not to resolve every dimension of the pay gap, but to make progress in identifying which factors matter most and how they interact in practice.