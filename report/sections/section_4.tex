\section*{Section 4: Event Study - Wage Changes around Moves}
\addcontentsline{toc}{section}{Section 4: Event Study - Wage Changes around Moves}
\label{sec:Section 4: Event Study - Wage Changes around Moves}

Observing how workers move between different coworker environments over time probes whether coworker wages are acting as structural inputs to wage determination or as signals of deeper, unobserved differences across workers.

By comparing wage paths for workers who transition across coworker-wage terciles, we can observe whether changes in coworker environments are associated with systematic changes in workers' own wages, and whether these patterns align more closely with a sorting-based or ability-based explanation.

\begin{figure}[H]
\centering
\includegraphics[width=\textwidth]{figures/fig3_event_study_ankit.png}
\caption{Event study of mean log wages for job changers, plotted from three years before to two years after the move. Each panel corresponds to the worker's initial coworker-wage tercile (T1, T2, T3). Within each panel, lines indicate the tercile of the coworker-wage environment at the \emph{second} job. The vertical dashed line marks the job transition.}
\label{fig:eventstudy}
\end{figure}


Figure~\ref{fig:eventstudy} compares wage trajectories for movers, grouped by where they start in the coworker-wage distribution (\texttt{owage1} tercile) and where they land after moving (\texttt{owage2} tercile). Across the panels where workers move up to higher coworker-wage terciles, there is a clear upward shift in mean log wages at event time $0$ (the job change) followed by a gradual upward trajectory afterward, especially for moves to End T3. Conversely, moves that stay within the same tercile show gradual wage growth with no large break at the transition and moves that progress down to lower coworker-wage terciles show a drop at event time $0$.

These patterns provide more support for Model 1 than Model 2. Given the strongest change happens right at the move, the graphs suggest that switching into a higher-coworker-wage workplace is a consequence of shorter-term mechanisms such as luck, connections, or search, rather than a long-running upward trajectory. It is expected that if Model 2 were the main mechanism, movers who end up with high-paid coworkers already display steeper wage growth before the move since model 2 correlates higher pay with higher skills and ambition. However, the graphs do not rule out Model 2 entirely as selection on unobserved ability could still matter if higher-skill workers are the ones who are able to access high-coworker-wage jobs, and the move could occur due to mechanisms such as promotion. Nevertheless, the event‐study evidence suggests that access to higher-coworker‐wage workplaces, rather than gradual skill accumulation, plays the dominant role in driving wage gains which provides stronger support for Model 1.

Having seen in Figure~\ref{fig:eventstudy} that wage jumps tend to accompany moves into higher coworker–wage environments, we next ask how much of this relationship survives once we difference out time–invariant worker attributes. To do so, we model the change in log wages from period $-1$ to period $0$,
\[
  \Delta y_i \equiv y_i - y_{i,-1},
\]
as a function of the corresponding change in coworkers' mean log wages,
\[
  \Delta \text{owage}_i \equiv \text{owage2}_i - \text{owage1}_i.
\]
Table~\ref{tab:table4} reports pooled first–difference regressions of $\Delta y_i$ on $\Delta \text{owage}_i$, first without additional controls and then adding education dummies, a quadratic in experience as of period $-1$, and a female indicator. Comparing these estimates to the level regressions in Table~\ref{tab:table3} allows us to gauge what fraction of the large coworker–wage coefficients reflects sorting or fixed individual traits, and what fraction remains even after we remove time–invariant heterogeneity.


\begin{table}[H]
\centering
\caption{Wage Changes and Coworker Wage Growth}
\label{tab:table4}

\begin{threeparttable}
\small
\setlength{\tabcolsep}{3pt}
\renewcommand{\arraystretch}{1.3}

\begin{tabular}{lccccccccccc}
\toprule
 & Intercept & $\Delta$ow & Ed9 & Ed12 & Ed16 & exp & exp$^2$ & exp$^3$ & Female & R$^2$ & Adj.\ R$^2$ \\
\midrule
(1) $\Delta y$ 
 & 0.038 & 0.293 & -- & -- & -- & -- & -- & -- & -- & 0.108 & 0.108 \\
 & (0.002) & (0.006) &  &  &  &  &  &  &  &  &  \\
\midrule
(2) $\Delta y$ 
 & 0.116 & 0.296 & 0.004 & 0.030 & 0.081 
 & -0.013 & 0.001 & -0.000 & -0.025 & 0.130 & 0.129 \\
 & (0.035) & (0.006) & (0.006) & (0.006) & (0.007) 
 & (0.007) & (0.000) & (0.000) & (0.004) &  &  \\
\bottomrule
\end{tabular}

\begin{tablenotes}[para,flushleft]
\footnotesize
\textit{Notes}: Each row reports a pooled OLS regression for the change in log wages 
$\Delta y_i = y_{i0} - y_{i,-1}$ between period $-1$ and period $0$. 
The regressor $\Delta$ow is the change in coworkers' mean log wage $(\text{owage2}_i - \text{owage1}_i)$. Ed9, Ed12, and Ed16 denote education dummies for 9, 12, and 16 years of schooling; the omitted category is 6 years. \textit{exp}, \textit{exp}$^2$, and \textit{exp}$^3$ are a cubic polynomial in experience at period $-1$. \textit{Female} is an indicator for women. Standard errors are in parentheses. R$^2$ is the coefficient of determination; Adj.\ R$^2$ is the adjusted R$^2$. Sample size is $N = 16{,}969$ in both specifications.
\end{tablenotes}

\end{threeparttable}
\end{table}



Table~\ref{tab:table4} shows that controlling for unobserved characteristics of people yields that changes in individual wages are strongly linked to changes in coworker wages across all specifications. In the first pooled model, a one log point rise in coworker wages raises an individual’s log wage by about $0.29$ points, while women’s wage growth is almost $2$\%  lower than that of men. In our following model, we add controls for experience and notice that the effect of coworker’s wages remain about the same suggesting that differences in experience barely contribute to explaining coworker wage effect. In the third model, which includes an interaction between gender and coworker wage growth, the coworker wage effect for men is approximately $0.3321$ log points while the interaction term having a coefficient of $-0.114$ means that for women, the coworker wage effect is roughly $0.22$ log points. 

We confirm this observation on conducting separate regressions by gender where we observe that the coefficient on coworker-wage growth is $0.332$, while for women it is $0.218$. This indicates that men’s wages are much more responsive to movements in the wages of their co workers. On comparing to the OLS levels results form table~\ref{tab:table3}, where the coworker wage coefficients were around $0.66$ for men and about $0.60$ for women; we interpret that about $50$\% ($0.503$) of the coworker wage coefficient for men and about $36$\% ($0.364$) of the coworker wage coefficient for women persist even after differencing out the unobserved individual factors. 
This difference suggests that about half of the relationship seen in the OLS results comes from sorting or unobserved traits. It is only the other half that reflects a real causal effect, which suggests that when people move to jobs with higher paid coworkers, usually their wages increase as well. The comparatively smaller effect for women indicates that women’s wages respond less to changes in coworker pay, likely due to unobserved barriers that women face at the workplace. Overall, coworker wages directly affect wage growth, and men benefit more from these workplace effects than women in terms of wages.
